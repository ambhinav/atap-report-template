\section{Knowledge And Experience Gained}
\subsection{Technical Knowledge Gained From Assignments}
\noindent
Firstly, I learnt about creating web-based platforms that are not only functional, but efficient as well. Through the assignment I was exposed to advanced web development concepts such as centralised state-management and DOM manipualtion. As the company was paying Google for use of tools such as Firebase, it was important that
CRUD opertations were optimised and re-rendering of components was minimised so as to reduce reads. Such financial considerations helped me gain exposure to important web architecture concepts which I would have otherwise
not come accross. This included improvements in my functional and asynchronous Javascript programming skills.

\noindent
Secondly, I gained a deeper understanding of Software Design Patterns and best practices. Having taken CS2103T in the previous semester, I was able to apply Software Engineering principles such as Seperation of Concerns and
Architectures such as Model View Controller in a more advanced setting. This improved the readability of my code and made debugging much easier, which was crucial to meet Sprint deadlines.

\subsection{Organizational/Industry Experience Gained From Assignments}
\noindent 
Firstly, I learnt more about the inner workings of a start-up and the way they are managed. Due to the size of start-ups such as Intra Technologies, employees are expected to have a working knowledge of all facets of the
company and be flexible to change in reporting structure. Nevertheless, it is easy
to approach high-level management such as the CEO to clarify issues or gain more knowledge. 

\noindent
For example, I was able to learn more about the branches of a company other than the Technological side. I was exposed to how
business development strategies such as marketing, roadmaps and product plans are created and the considerations that are taken into producing them.

\noindent
Next, in these first three months, I was exposed to the Planning, Defining, Designing and Building stages of the Software Development Lifecycle. Also, for the building of the software the company decided to adopt Scrum techniques that
helped improve my productivity and the quality of code I wrote.


\subsection{Areas of Applicability of Knowledge And Experienced Gained}
\noindent
On the technical side, Full-Stack Software Development is a skillset that is increasingly vital to industries that are digitalizing. Learning the Software Architecture and how the different layers interact helps me understand how to design each layer and the overall
Architecture easily for any use case I may encounter in my career.

\noindent
On the non-technical side, I believe that working in a start-up will help give me a leg up in understanding businesses strategies in my future workplace. Furthermore, I am more knowledgable about branding and product development strategies. Such knowledge may help if I decide to expand my career
in that direction.
